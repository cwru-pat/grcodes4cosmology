\documentclass[english]{article}
\usepackage[utf8]{inputenc}
\usepackage{geometry}
\geometry{verbose,tmargin=1in,bmargin=1in,lmargin=1in,rmargin=1in}
\usepackage{esint}
\usepackage{babel}
\begin{document}

\section{Counter-Streaming Velocity Profile Initial Conditions}

In a 3+1/ADM split, the metric is written as
\begin{equation}
g_{\mu\nu}=\left(\begin{array}{cc}
-\alpha^{2}+\gamma_{kl}\beta^{k}\beta^{l} & \gamma_{ik}\beta^{k}\\
\gamma_{jk}\beta^{k} & \gamma_{ij}
\end{array}\right)
\end{equation}
and
\begin{equation}
g^{\mu\nu}=\left(\begin{array}{cc}
-\alpha^{-2} & \alpha^{-2}\beta^{i}\\
\alpha^{-2}\beta^{j} & \gamma^{ij}-\alpha^{-2}\beta^{i}\beta^{j}
\end{array}\right)\,.
\end{equation}


\subsection{The Constraint Equations}

On the initial surface, the Hamiltonian and Momentum constraints must
be satisfied. In terms of ADM variables, these are respectively given
by
\begin{eqnarray}
^{(3)}R+K^{2}-K_{ij}K^{ij} & = & 16\pi\rho\\
D_{j}(K^{ij}-\gamma^{ij}K) & = & S^{i}
\end{eqnarray}
For a pressureless, perfect fluid with rest-density $\rho_{0}$ and
some non-zero velocity $U_{\mu}$, the stress-energy tensor is 
\begin{equation}
T_{\mu\nu}=\rho_{0}U_{\mu}U_{\nu}\,.
\end{equation}
The covariant source terms for the $3+1$ momentum constraint equations
are then 
\begin{eqnarray}
S^{i} & = & -\gamma^{ij}n^{\mu}T_{j\mu}\\
 & = & \alpha\rho_{0}\gamma^{ij}U_{j}U^{t}\,,
\end{eqnarray}
and the Hamiltonian constraint source density is 
\begin{eqnarray}
\rho & = & n_{\mu}n_{\nu}T^{\mu\nu}\\
 & = & \alpha^{2}\rho_{0}(U^{t})^{2}\,,
\end{eqnarray}
given the normal to the spatial slices is $n^{\mu}=\frac{1}{\alpha}(1,-\beta^{i})$
or $n_{\mu}=(-\alpha,0)$. 


\subsection{Initial Conditions}

We can attempt to come up with a solution to these equations given
an initally flat metric, with $\gamma_{ij}=\delta_{ij}$, and for
constant trace of the extrinsic curvature $K$. (A variable conformal
factor could be used for the 3-metric, but for simplicity this is
taken to be unity on the initial surface.) To obtain a solution to
the momentum constraint, a traceless-type decomposition can be performed,
where the extrinsic curvature $K_{ij}$ is written as a trace-free
part $A_{ij}$ (so that $\gamma_{ij}A^{ij}=0$) and traceful part,
\begin{equation}
K_{ij}=A_{ij}+\frac{1}{3}\gamma_{ij}K\,.
\end{equation}
In terms of these variables and for the choice of metric variables
so far, the momentum constraint equation reduces to
\begin{equation}
\partial_{j}A^{ij}=S^{i}\,.
\end{equation}
As a simple example, a single component source $S^{i}\sim(S^{x}(y),0,0)$,
which happens when $U_{j}=(U_{x}(y),0,0)$, results in only a single
non-zero constraint equation, 
\begin{eqnarray}
\partial_{j}A^{xj} & = & S^{x}(y)\,.
\end{eqnarray}
The most straightforward solution to this is $A^{xy}=A^{yx}=\int dyS^{x}(y)$
with all other components of $A^{ij}$ zero. In terms of $A_{ij}$,
the Hamiltonian constraint is 
\begin{eqnarray}
\frac{2}{3}K^{2}-A_{ij}A^{ij} & = & 16\pi\rho\,,
\end{eqnarray}
so $\rho$ can be determined algebraically. The primitive fluid variables
$\rho_{0}$ and $U_{x}$ can then be solved for,
\begin{eqnarray}
U_{x} & = & \frac{Q}{(1-Q)},\,\,{\rm with}\\
Q & = & \left[\frac{16\pi S^{x}}{\left(\frac{2}{3}K^{2}-\left[\int dyS^{x}(y)\right]^{2}\right)}\right]^{2}
\end{eqnarray}
and
\begin{equation}
\rho_{0}=\frac{S^{x}}{U_{x}\sqrt{1+U_{x}^{2}}}\,.
\end{equation}

In summary, for this solution, the momentum variable $S^{x}(y)$ and a
constant (homogeneous) value for metric variable $K$ are freely chosen.
To mimic an expanding cosmology, $K$ should be negative and large enough
that $\rho>0$. The 3-metric is initially $\gamma_{ij}=\delta_{ij}$, and
$A_{ij}$ is $A^{xy}=A^{yx}=\int dyS^{x}(y)$. The source density $\rho$
is determined algebraically from the Hamiltonian constraint, and fluid
primitives from $S^{x}$ and $\rho$.
\end{document}
